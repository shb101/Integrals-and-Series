\section{Beta Function}
The Beta Function is defined by 
\begin{align*}
B(x,y) &= \int_0^1 t^{x-1}(1-t)^{y-1} \; dt
\end{align*}
for $\text{Re }x > 0$, $\text{Re }y>0$. It is related to the Gamma function by
\begin{align*}
B(x,y) = \frac{\Gamma(x)\Gamma(y)}{\Gamma(x+y)}
\end{align*}
\subsection{Integral 1}
Prove that
\begin{align*} \int_0^1 x^{a-1}(1-x)^{b-1} \frac{dx}{(x+p)^{a+b}} = \frac{\Gamma(a)\Gamma(b)}{\Gamma(a+b)}\frac{1}{(1+p)^{a} p^b}\end{align*}
where $a,\; b, \; p   >0$.
\begin{proof}
Perform the change of variables $\displaystyle \frac{x}{p+x}=\frac{t}{p+1}$. Then
\begin{align*}\frac{p}{(p+x)^{2}}&= \frac{1}{p+1}\frac{dt}{dx} \\ \Rightarrow \frac{dx}{(p+x)^{2}} &=\frac{dt}{p(p+1)}\end{align*}
After making the substitutions, the integral transforms into:
\begin{align*}
 \int_0^1 x^{a-1}(1-x)^{b-1} \frac{dx}{(x+p)^{a+b}} &= \frac{1}{p^b (1+p)^a}\int_0^1 t^{a-1}(1-t)^{b-1} dt \\
&= \frac{1}{p^b (1+p)^a} \frac{\Gamma(a)\Gamma(b)}{\Gamma(a+b)}
\end{align*}
\end{proof}

\subsection{Integral 2}
\begin{align*}
\int_{0}^{1}\frac{1}{(2-x)\sqrt[5]{x^{2}(1-x)^{3}}}dx = \frac{2\pi \sqrt[10]{2}}{\sqrt{5+\sqrt{5}}}
\end{align*}
\begin{proof}
Perform the change of variables $t=1-x$. Then, 
\begin{align*}
\int_0^1 \frac{1}{(2-x)\sqrt[5]{x^2(1-x)^3}}dx &= \int_0^1 \frac{1}{(1+t)\sqrt[5]{(1-t)^2 t^3}}dt
\end{align*}
We now have a special case of Integral 1 with $a=\frac{2}{5}, \; b=\frac{3}{5}$ and $p=1$. Therefore,
\begin{align*}
\int_0^1 \frac{1}{(2-x)\sqrt[5]{x^2(1-x)^3}}dx  &= \frac{ \Gamma \left(\frac{2}{5} \right)\Gamma \left(\frac{3}{5} \right)}{2^{\frac{2}{5}}}\\
&= \frac{\pi}{2^{\frac{2}{5}} \sin \left( \frac{2\pi}{5}\right)} \\
&= \frac{2\pi \sqrt[10]{2}}{\sqrt{5+\sqrt{5}}}
\end{align*}
\end{proof}
